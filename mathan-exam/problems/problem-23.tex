Дайте определения эквивалентных функций, бесконечно малой функции и функций одного порядка. Найдите функцию $g(x) = Cx^a$, эквивалентную функции $f(x)= \sqrt[3]{x^6+3\sqrt[5]{x}}$, при х -> 0, x -> inf.

\subsection*{Решение}

\begin{defi} бесконечно малые функции a(x) и b(x) называются эквивалентными бесконечно малыми при х->a, если
$\displaystyle \lim_{n\rightarrow a}
               \frac{a(x)}{b(x)}
               =1
$ 
\end{defi}
\begin{figure}[h!]
\includegraphics[width=1\linewidth]{_PNG}
\end{figure}
\clearpage
\begin{defi}  функции a(x) и b(x) называются бесконечно малыми одного порядка малости при х->a, если
$\displaystyle \lim_{n\rightarrow a} 
               \frac{a(x)}{b(x)}
               =c, c != 0
$              

\end{defi}
\begin{defi} функция y = f(x) называется бесконечно малой при x -> a, если
$\displaystyle \lim_{n\rightarrow a} 
               f(x) = 0
$
             
\end{defi}

\textbf{Примеры:}
    \begin{enumerate}
        \item $f(x) = (x-1)^2$ -- бесконечно малая при х -> 1.
         \item $f(x) = \frac{1}{x}$ -- бесконечно малая при х -> $inf$.
    \end{enumerate}


    \subsection*{Задание}
    Найдите функцию $g(x) = Cx^a$, эквивалентную функции $f(x)= \sqrt[3]{x^6+3\sqrt[5]{x}}$, при х -> 0, x -> inf.
        \begin{enumerate}
            \item x -> 0. Рассмотрим g(x) = $\sqrt[3]{3}\cdot\sqrt[15]{x}$
            
            $\displaystyle \lim_{n\rightarrow 0} 
               \frac{
                   \sqrt[3]{x^6+3\cdot \sqrt[5]{x}}
               }
               {
               \sqrt[3]{x^6+3\sqrt[5]{x}}
               }
               =
               \displaystyle \lim_{n\rightarrow 0} 
               \frac{
                   \sqrt[3]{\frac{x^6}{x^{1/5}} + 3 \cdot \frac{x^{1/5}}{x^{1/5}}}
               }
               {
               \sqrt[3]{3}
               }
               =
               \displaystyle \lim_{n\rightarrow 0} 
               \frac{
                   \sqrt[3]{x^{29/5} + 3}
               }
               {
               \sqrt[3]{3}
               }
               = 
               1
$    
            \item Это значит, что g(x) - искомая эквивалентная функция.
            
            \item x -> inf, положим g(x) = $x^2$
            $\displaystyle \lim_{n\rightarrow \infty} 
               \frac{
                   \sqrt[3]{x^6+3\cdot \sqrt[5]{x}}
               }
               {
               x^2
               }
               =
               \displaystyle \lim_{n\rightarrow \infty} 
                   \sqrt[3]{1 + 3 \cdot x^{-29/5}}
               = 
               1
$    
    \item А значит, g(x) - искомая эквивалентная функция. Мы красавы!
        \end{enumerate}
