Сформулируйте теорему о существовании точных верхней и нижней граней и приведите \textit{схему ее доказаиельства}. Докажите утверждение, пользуясь методом математической индукции

\subsection{}
	Теорема об отделимости:\\
	Пусть $$X \subset R, Y\subset \mathbb{R}, X \neq \emptyset, Y\neq\emptyset;$$
	$$\forall x\in X, y\in Y: x\leq y.$$
	Тогда $$\exists supX, infY: \forall x\in X, y\in Y: x\leq supX\leq infY\leq y.$$
	\\\\
\subsection{}
	Доказать по индукции: $$\prod_{i=1}^n\frac{2i - 1}{2i}\leq\frac{1}{\sqrt{3k+1}}$$
	База $n = 1$:
	$$\frac{1}{2}\leq\frac{1}{\sqrt{3+1}}; \frac{1}{2}\leq\frac{1}{2}$$
	Предположим, что $\prod_{i=i}^{k}\frac{2i - 1}{2i}\leq\frac{1}{\sqrt{3k+1}}$ верно для некоторого k. Для k+1 выражение будет иметь вид:
	$$\prod_{i=i}^{k}\frac{2i - 1}{2i}\cdot\frac{2k + 1}{2k+2}\leq\frac{1}{\sqrt{3k+4}};$$
	$$\prod_{i=i}^{k}\frac{2i - 1}{2i}\leq\frac{2k + 2}{(2k+1)\sqrt{3k+4}};$$
	Чтобы из предположения индукции следовала верность утверждения выше, докажем еще один факт:
	$$\frac{1}{\sqrt{3k+1}}\leq\frac{2k + 2}{(2k+1)\sqrt{3k+4}};$$
	$$\frac{2k+1}{2k+1}\cdot\sqrt{\frac{3k + 1}{3k+4}}\geq 1;$$
	$$\frac{(2k+1)^2}{(2k+1)^2}\cdot\frac{3k + 1}{3k+4}\geq 1;$$
	$$\frac{(4k^2+8k+4)(3k+1)}{(4k^2+4k+1)(3k+4)}\geq 1;$$
	$$12k^3+28k^2+20k+4\geq 12k^3+28k^2+19k+4;$$
	$$20k\geq 19k;$$
	\\\\
\subsection{}
	Здесь приведены также аксиомы, которые возможно не требуются в задании.
	Арифметические операции на $\mathbb{R}$:\\
	Сложение $\mathbb{R}\times\mathbb{R}\xrightarrow{+}\mathbb{R}$:\\
	1. $\exists 0: \forall x\in\mathbb{R}:x+0=0+x=x;$\\
	2. $\forall x\in\mathbb{R} \exists (-x)\in\mathbb{R}:(-x)+x=x+(-x)=0;$\\
	3. $\forall x, y, z\in \mathbb{R}: (x + y) + z = x + (y + z)$ -- ассоциативность;\\
	4. $\forall x, y: x + y = y + x$ -- коммутативность;\\\\
	Умножение $\mathbb{R}\times\mathbb{R}\xrightarrow{\cdot}\mathbb{R}$\\
	1. $\exists 1: \forall x\in\mathbb{R}: 1\cdot x=x\cdot 1 = x;$\\
	2. $\forall x\in\mathbb{R}\setminus\{0\}\exists x^{-1}: x\cdot x^{-1} = x^{-1}\cdot x = 1;$\\
	3. $\forall x, y, z\in\mathbb{R}:(xy)z=x(yz)$ -- ассоциативность;\\
	4. $\forall x, y\in\mathbb{R}: xy=yx$ -- коммутативность;\\
	5. $\forall x, y, z\in\mathbb{R}: x(y+z) = xy+xz$ -- дистрибутивность;\\
