\section{}
Сформулируйте теорему о существовании точных верхней и нижней граней и приведите \textit{схему ее доказаиельства}. Докажите утверждение, пользуясь методом математической индукции

\subsection{}
	Теорема об отделимости:\\
	Пусть $$X \subset R, Y\subset \mathbb{R}, X \neq \emptyset, Y\neq\emptyset;$$
	$$\forall x\in X, y\in Y: x\leq y.$$
	Тогда $$\exists supX, infY: \forall x\in X, y\in Y: x\leq supX\leq infY\leq y.$$
	\\\\
\subsection{}
	Доказать по индукции: $$\prod_{i=1}^n\frac{2i - 1}{2i}\leq\frac{1}{\sqrt{3k+1}}$$
	База $n = 1$:
	$$\frac{1}{2}\leq\frac{1}{\sqrt{3+1}}; \frac{1}{2}\leq\frac{1}{2}$$
	Предположим, что $\prod_{i=i}^{k}\frac{2i - 1}{2i}\leq\frac{1}{\sqrt{3k+1}}$ верно для некоторого k. Для k+1 выражение будет иметь вид:
	$$\prod_{i=i}^{k}\frac{2i - 1}{2i}\cdot\frac{2k + 1}{2k+2}\leq\frac{1}{\sqrt{3k+4}};$$
	$$\prod_{i=i}^{k}\frac{2i - 1}{2i}\leq\frac{2k + 2}{(2k+1)\sqrt{3k+4}};$$
	Чтобы из предположения индукции следовала верность утверждения выше, докажем еще один факт:
	$$\frac{1}{\sqrt{3k+1}}\leq\frac{2k + 2}{(2k+1)\sqrt{3k+4}};$$
	$$\frac{2k+1}{2k+1}\cdot\sqrt{\frac{3k + 1}{3k+4}}\geq 1;$$
	$$\frac{(2k+1)^2}{(2k+1)^2}\cdot\frac{3k + 1}{3k+4}\geq 1;$$
	$$\frac{(4k^2+8k+4)(3k+1)}{(4k^2+4k+1)(3k+4)}\geq 1;$$
	$$12k^3+28k^2+20k+4\geq 12k^3+28k^2+19k+4;$$
	$$20k\geq 19k;$$
	\\\\
\subsection{}
	Здесь приведены также аксиомы, которые возможно не требуются в задании.
	Арифметические операции на $\mathbb{R}$:\\
	Сложение $\mathbb{R}\times\mathbb{R}\xrightarrow{+}\mathbb{R}$:\\
	1. $\exists 0: \forall x\in\mathbb{R}:x+0=0+x=x;$\\
	2. $\forall x\in\mathbb{R} \exists (-x)\in\mathbb{R}:(-x)+x=x+(-x)=0;$\\
	3. $\forall x, y, z\in \mathbb{R}: (x + y) + z = x + (y + z)$ -- ассоциативность;\\
	4. $\forall x, y: x + y = y + x$ -- коммутативность;\\\\
	Умножение $\mathbb{R}\times\mathbb{R}\xrightarrow{\cdot}\mathbb{R}$\\
	1. $\exists 1: \forall x\in\mathbb{R}: 1\cdot x=x\cdot 1 = x;$\\
	2. $\forall x\in\mathbb{R}\setminus\{0\}\exists x^{-1}: x\cdot x^{-1} = x^{-1}\cdot x = 1;$\\
	3. $\forall x, y, z\in\mathbb{R}:(xy)z=x(yz)$ -- ассоциативность;\\
	4. $\forall x, y\in\mathbb{R}: xy=yx$ -- коммутативность;\\
	5. $\forall x, y, z\in\mathbb{R}: x(y+z) = xy+xz$ -- дистрибутивность;\\
\section{}
Дайте определение равномозных множеств, счетного множества. \textit{Докажите}, что счетное объединение счетных множеств счетно. Покажите, что $(0,1] $ равномощно $ [0, 1]$.
\subsection{}
Множества называются равномощными, если существует биекция из одного множества в другое.
\subsection{}
Множество называется счетным, если оно равномощно $\mathbb{N}$
\subsection{}
Счетное объединение -- объединение счетного множества множеств.\\
Пусть есть счетное множество множеств $A_0, A_1, A_2 ... $. Тогда составим такую таблицу, что в i-й строке на j-м месте будет стоять элемент i-го множества(множеств счетное количество -- можем занумеровать их) с номером j(все множества счетны -- можем занумеровать элементы внутри них). Этот элемент обозначим за $a_{ij}$. Далее выделим в этой таблице <<диагонали>>, т.е. последовательности элементов, сумма индексов которых равна. Нулевая диагональ -- левый верхний элемент $a_{00}$; первая диагональ, элементы $a_{01}$ и $a_{10}$; вторая -- $a_{02}$, $a_{11}$, $a_{20}$... Далее будем строить последовательность всех элементов всех множеств. Будем брать диагонали по возрастанию суммы индексов в них и дописывать в конец последовательности(в самом начале пустой) числа этой диагонали в порядке возрастания номера строки(= убывания номера столбца). При этом в случае, если очередной выписываемый элемент уже встретился и был выписан ранее, то не будем его выписывать.\\
Ясно, что в итоге получится счетное множество, т.к. счетно как каждое из множеств, так и множество всех множеств. Ч.т.д.
\subsection{}
Из $(0;1]$ выделим в множество $A$ бесконечную убывающую последовательность: $\{\frac{1}{1}, \frac{1}{2},\frac{1}{3}, ...\}$. Останется $(0;1]\setminus A$. Из $[0;1]$ выделим в множество $A`$ сперва 0, а затем точно такую же последовательность, как выделили из $(0;1]$. Останется $[0;1]\setminus A`$. Заметим, что после выделения $A$ и $A`$ и там и там остались одинаковые множества. Биекция между ними очевидна. Осталось провести биекцию между выделенными последовательностями:
$$A = \{\frac{1}{1}, \frac{1}{2}, \frac{1}{3}, \frac{1}{4}, ...\},$$
$$A` = \{0, \frac{1}{1}, \frac{1}{2}, \frac{1}{3}, ...\}.$$
Просто сопоставим элементы этих множеств таким образом: $\frac{1}{1}\leftrightarrow 0, \frac{1}{2}\leftrightarrow \frac{1}{1}, \frac{1}{3}\leftrightarrow \frac{1}{2}, \frac{1}{4}\leftrightarrow \frac{1}{3} ...$
\section{}
\subsection{}
(этот пункт до конца не хватило времени доказать, но гуглится легко)0 )\\
Чтобы показать счетность $\mathbb{Q}$, покажем сперва счетность множества неотрицательных рациональных чисел, затем скажем, что для отрицательных рассуждение аналогичное, и что из счетности этих двух множеств вытекает, что их объединение так же счетно(пруф -- 5.в для случая с всего двумя множествами).\\
Неотрицательное рациональное число представимо в виде пары чисел -- натурального или нулевого числителя и натурального ненулевого знаменателя. Получается, есть биекция из множества положительных рациональных чисел в декартово произведение ...
\subsection{}
Чтобы показать несчетность $\mathbb{R}$, достаточно показать несчетность (0;1).\\
Предположим, интервал счетен и мы занумеровали все числа в нём каким-то образом. Тогда их можно выписать в таком виде:\\
$$x_1 = \overline{0.a_{11}a_{12}a_{13}a_{14}a_{15}...}$$
$$x_2 = \overline{0.a_{21}a_{22}a_{23}a_{24}a_{25}...}$$
$$x_3 = \overline{0.a_{31}a_{32}a_{33}a_{34}a_{35}...}$$
$$...$$
(тут $a_{ij}$ -- j-ая цифра после точки в i-ом числе)\\
Ясно, что все эти числа принадлежат указанному интервалу, кроме тех, где все цифры нули или девятки.\\
Построим теперь такое число $x$, которое тоже будет принадлежать интервалу, но будет отлично от любого из тех, что нам удалось занумеровать и выписать выше:
$$x = \overbrace{0.b_{11}b_{22}b_{33}b_{44}b_{55}...},$$
где $b_{ii} \neq a{ii}, b_{ii} \neq 0, b_{ii} \neq 9$. Тогда получается, что $x$ отличен от $x_i$ в i-й позиции после точки, т.е. отличен от любого из чисел. Значит, как бы мы не нумеровали числа этого интервала, в нем всегда можно будет найти незанумерованное.
\subsection{}
Множество, состоящее из десятичных дробей с нулевой целой частью и дробной, составленной из нулей и единичек не будет счетно по аналогии с несчетностью $\mathbb{R}$.\\
Точно так же предположим, что смогли занумеровать все числа вида, описанного в задаче, далее составим еще одно число, отличающееся от всех занумерованных в какой-то одной позиции. То есть как бы мы не нумеровали все такие числа, всегда найдется незанумерованное число.
