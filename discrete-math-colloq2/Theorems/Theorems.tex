\documentclass[a4paper,12pt]{article}

\usepackage{header}

\begin{document}
	\title{Дискретная математика. Коллоквиум весна 2017. \\ Теоремы}
	\author{Орлов Никита}
	\maketitle
	
	\section*{Теорема 6}
	\begin{theorem}
		Всякое подмножество $B$ счетного множества $A$ конечно или счетно.
	\end{theorem}
	\begin{proof}
		Выпишем элементы множества $A$ в последовательность:
		\[
		A = \{a_0, a_1, \ldots, a_n, \ldots \}
		\]
		
		Будем вычеркивать из этой последовательности элементы, не лежащие в $B$. В итоге останется конечная либо счетная последовательность. 
	\end{proof}

	\begin{theorem}
		Любое бесконечное множество содержит счетное подмножество.
	\end{theorem}
	\begin{proof}
		Для бесконечного множества $A$ будем выделять счетное подмножество. Первый элемент $b_0 \in A$ выберем произвольно. Затем рассматриваем множество $A \setminus \{b_0\}$. Оно бесконечно, а значит мы можем выбрать новый элемент $b_1$. Утверждается, что этот процесс мы можем проделать бесконечное количество раз: на каждом $i$ шаге мы остаемся с множеством $A \setminus \{b_o \ldots b_i\}$, которое бесконечно.
	\end{proof}
    
	\sep	
	
	
	\section*{Теорема 8}
	
	\begin{theorem}
		Декартово произведение счетных множеств счетно.
	\end{theorem}
	\begin{proof}
		Б.о.о. можно считать, что необходимо доказать счетность $\N \times \N$. Разобьем наше декартово произведение в объединение множеств вида $ \{a_0\} \times \N $. Каждое такое множество счетно. В итоге декартово произведение разложилось в счетное объединение счетных множеств, а значит и само счетно.
	\end{proof}
	
	\section*{Теорема 9}
	\begin{theorem}
		Множество бесконечных последовательностей нулей и единиц несчетно. 
	\end{theorem}

	\begin{proof}
		Предположим, что оно счетно, значит его можно пронумеровать. Тогда построим таблицу
		последовательностей.
		\begin{center}
			\begin{tabular}{cccc}
				$a_{00}$ & $a_{01}$ & $a_{02}$ & $\ldots$ \\
				$a_{10}$ & $a_{11}$ & $a_{12} ...$ \\
				$a_{20} a_{21} a_{22} ...$ \\
			\end{tabular}
		\end{center}
		
		
		Теперь рассмотрим диагональную последовательность  $a_{00} a_{11} a_{22} ...$ и
		заменим в ней все биты на противоположные. Такая последовательность отличается
		от любой $a_{i}$ в i-й позиции, значит этой последовательности нет в списке,
		получили противоречие. Значит это множество несчетно.
		Теперь докажем, что множество бесконечных последовательностей нулей и единиц равномощно
		отрезку [0;1], то есть имеет мощность континуум.
		Из курса анализа известно, что каждое число из [0;1] можно представить в виде
		бесконечной двоичной дроби. Делается это так: первый бит после запятой равен 0, если
		x лежит в левой половине отрезка [0,1] и равен 1, если в правой. И так далее. Делим
		отрезок пополам и смотрим, куда попал x. Но это не совсем биекция. Такие последовательности
		как 0,1001111... и 0, 101000... соотвествуют одному и тому же числу. Чтобы исправить
		это, надо исключить последовательности, в которых начиная с некоторошо момента все
		цифры равны 1 (кроме 0.111111...). Но таких последовательностей счетное множество, так
		что их добавление не меняет мощность множества.
	\end{proof}

	
	\sep	
	
	\section*{Задача 4}

	\subsection*{Решение}


	\sep	
	
	\section*{Задача 5}
	
	\subsection*{Решение}
	
	

	\sep	
		
		
	\section*{Задача 6}

	
	\subsection*{Решение}
	


	\sep		
	
	
	
	
	\section*{Задача 7}

	
	\subsection*{Решение}
	


	\sep	
	
	
	
	
\end{document}
