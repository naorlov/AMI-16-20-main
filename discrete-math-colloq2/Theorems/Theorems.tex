\documentclass[a4paper,12pt]{article}

\usepackage{header}
%\setcounter{secnumdepth}{0} % sections are level 1


\begin{document}
	\title{Дискретная математика. Коллоквиум весна 2017. \\ Теоремы}
	\author{Орлов Никита}
	\maketitle
	\tableofcontents
    \pagebreak
	\section{Теорема 6}
	\begin{theorem}
		Всякое подмножество $B$ счетного множества $A$ конечно или счетно.
	\end{theorem}
	\begin{proof}
		Выпишем элементы множества $A$ в последовательность:
		\[
		A = \{a_0, a_1, \ldots, a_n, \ldots \}
		\]
		
		Будем вычеркивать из этой последовательности элементы, не лежащие в $B$. В итоге останется конечная либо счетная последовательность. 
	\end{proof}

	\begin{theorem}
		Любое бесконечное множество содержит счетное подмножество.
	\end{theorem}
	\begin{proof}
		Для бесконечного множества $A$ будем выделять счетное подмножество. Первый элемент $b_0 \in A$ выберем произвольно. Затем рассматриваем множество $A \setminus \{b_0\}$. Оно бесконечно, а значит мы можем выбрать новый элемент $b_1$. Утверждается, что этот процесс мы можем проделать бесконечное количество раз: на каждом $i$ шаге мы остаемся с множеством $A \setminus \{b_o \ldots b_i\}$, которое бесконечно.
	\end{proof}
    
	\sep	
	
	
	\section{Теорема 8}
	
	\begin{theorem}
		Декартово произведение счетных множеств счетно.
	\end{theorem}
	\begin{proof}
		Б.о.о. можно считать, что необходимо доказать счетность $\N \times \N$. Разобьем наше декартово произведение в объединение множеств вида $ \{a_0\} \times \N $. Каждое такое множество счетно. В итоге декартово произведение разложилось в счетное объединение счетных множеств, а значит и само счетно.
	\end{proof}
	
	\section{Теорема 9}
	\begin{theorem}
		Множество бесконечных последовательностей нулей и единиц несчетно. 
	\end{theorem}

	\begin{proof}
		Предположим, что оно счетно, значит его можно пронумеровать. Тогда построим таблицу
		последовательностей.
		\begin{center}
			\begin{tabular}{cccc}
				$a_{00}$ & $a_{01}$ & $a_{02}$ & $\ldots$ \\
				$a_{10}$ & $a_{11}$ & $a_{12} ...$ \\
				$a_{20} a_{21} a_{22} ...$ \\
			\end{tabular}
		\end{center}
		
		
		Теперь рассмотрим диагональную последовательность  $a_{00} a_{11} a_{22} ...$ и
		заменим в ней все биты на противоположные. Такая последовательность отличается
		от любой $a_{i}$ в i-й позиции, значит этой последовательности нет в списке,
		получили противоречие. Значит это множество несчетно.
		Теперь докажем, что множество бесконечных последовательностей нулей и единиц равномощно
		отрезку [0;1], то есть имеет мощность континуум.
		Из курса анализа известно, что каждое число из [0;1] можно представить в виде
		бесконечной двоичной дроби. Делается это так: первый бит после запятой равен 0, если
		x лежит в левой половине отрезка [0,1] и равен 1, если в правой. И так далее. Делим
		отрезок пополам и смотрим, куда попал x. Но это не совсем биекция. Такие последовательности
		как 0,1001111... и 0, 101000... соотвествуют одному и тому же числу. Чтобы исправить
		это, надо исключить последовательности, в которых начиная с некоторошо момента все
		цифры равны 1 (кроме 0.111111...). Но таких последовательностей счетное множество, так
		что их добавление не меняет мощность множества.
	\end{proof}

	
	\sep	
	\section{Теорема 16}
    \begin{theorem}
        Схема проверки связности графа на n вершинах полиномиального размера.
    \end{theorem}
    \begin{proof}
        Пусть матрица $A$ - матрица смежности графа с единицами на главной диагонали.
        Можно показать, что на пересечении строки $i$ и столбца $j$ матрицы $A^{k}$
        записано число путей длины $k$ из вершины $v_i$ в вершину $v_j$.
        Теперь рассмотрим матрицу $A'$, которая отличается от матрицы A тем, что
        у нее стоят единицы на главной диагонали.
        
        Заметим следующий факт: если между двумя вершинами есть путь длины меньше n - 1,
        то есть и путь длины
        ровно $n-1$, достаточно добавить нужное количество петель. То есть надо рассмотреть
        матрицу $(A')^{n-1}$. Если в яйчеках нет нулей - граф связен, иначе нет. Теперь
        опишем схему.
        
        На вход схема получает матрицу смежности $A'$. Схема последовательно
        вычисляет булевы степени этой матрицы $(A')^2,\ldots,(A')^{n-1}$. Затем
        схема вычисляет конъюнкцию всех ячеек матрицы $(A')^{n-1}$ и подает ее на выход.
        
        Оценим размер схемы. Для булева умножения достаточно $n^2\cdot O(n) = O(n^3)$
        операций. Всего нам нужо $(n-1)$ умножений, так что для вычисления матрицы $(A')^{n-1}$
        достаточно $O(n^4)$ операций. Для последнего этапа - конъюнкции нужно $O(n^2)$
        операций. Итого получается $O(n^4)+O(n^2) = O(n^4)$ операций.
        
    \end{proof}
    
    \section{Теорема 24}
    \begin{theorem}
    Непустое множество значений вычислимой функции является множеством значений всюду определенной вычислимой функции.    
    \end{theorem}
    \begin{proof}
        Пусть $S = f(\N)$ -- множество значений функции $f$. Так как $S$ непусто, зафиксируем некоторое $a \in S$. Пусть $U(p, x)$ -- некоторая универсальная фукнция, $p_0$ -- такое число, что $\forall x \ U(p_0, x) \equiv f(x)$. Пусть $F(p, x, t)$ -- отладочная функция для $U$.
        
        Опишем фукнцию $g : \N \times \N \rightarrow \N$:
        \[
        g(x, t) =
        \begin{cases}
            U(p_0, x), & F(p_0, x, t) = 1 \\
            a, & \text{иначе}
        \end{cases}
        \]
        
        Функция $g$ тотальна: если отладочная функция выдала 1, то универсальная функция тоже определена и остановится. Это следует из определения отладочной функции. Осталось показать, что множество всех значений $\{g(t, x) : t \in \N \ x \in \N \} = S$. В одну сторону: если $y = g(x, t)$, то $y = a \in S$ или $y = U(p_0, x) = f(x) \in S$. В другую: пусть $y = f(x) = U(p_0, x)$. На паре $(p, x)$ функция $U$ определена, а значит $\exists t: F(p_0, x, t) = 1$, но тогда $y = g(x, t)$. 
        
        Получили, что множество $S$ представимо в виде множества значений некоторой тотальной функции от двух аргументов. Осталось перейти к функции одного аргумента, использовав некоторую вычислимую биекцию $h: \N \rightarrow \N \times \N$. В итоге $S = (g \circ c)(\N)$.
        
        
    \end{proof}
	\sep	
	
	\section{Теорема 25}
	Пусть $U(p, x)$ -- универсальная функция. Рассмотрим множество $H$ такое, что
    \[
    H = \{ x : U(x, x) \text{ определена} \}
    \]
	
    \begin{theorem}
        Множество $H$ перечислимо, но не разрешимо.
    \end{theorem}
    \begin{proof}
        
        \
        
        \textit{Перечислимость.} Множество $H$ является областью определения вычислимой функции $x \mapsto U(x, x)$, а значит оно перечислимо.
        
        \textit{Неразрешимость.} Предположим, что $H$ разрешимо.
        
        Определим функцию $f$ следующим образом:
        \[
        f(x) = 
        \begin{cases}
            \uparrow, & x \in H \\
            1,        & x \notin H \\
        \end{cases}
        \]
        
        Пусть $p$ -- такое число, что $\forall x \ U(p, x) = f(x)$. Предположим, что $p \in H$. Тогда алгоритм вычисления $f$ не остановится, а значит $f(p) = U(p, p)$ не определено. По определению множества $H$ это означает, что $p \notin H$.
        
        Если же $p \notin H$, то алгориитм вычисления $f$ даст 1, то есть $1 = f(p) = U(p, p)$. Следовательно $p \in H$.
        
        Получили противоречие, а значит и множество $H$ неразрешимо.
    \end{proof}
	

	\sep	
		

	


	\sep		
	
	
	
	
	\section*{Задача 7}

	
	\subsection*{Решение}
	


	\sep	
	
	
	
	
\end{document}
