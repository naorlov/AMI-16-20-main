\documentclass[a4paper,12pt]{article}

\usepackage{header}

\begin{document}
	\title{Дискретная математика. <<ТЕМА>>}
	\author{Орлов Никита}
	\maketitle
	
	\section*{Теорема 6}
	\begin{theorem}
		Всякое подмножество $B$ счетного множества $A$ конечно или счетно.
	\end{theorem}
	\begin{proof}
		Выпишем элементы множества $A$ в последовательность:
		\[
		A = \{a_0, a_1, \ldots, a_n, \ldots \}
		\]
		
		Будем вычеркивать из этой последовательности элементы, не лежащие в $B$. В итоге останется конечная либо счетная последовательность. 
	\end{proof}

	\begin{theorem}
		Любое бесконечное множество содержит счетное подмножество.
	\end{theorem}
	\begin{proof}
		Для бесконечного множества $A$ будем выделять счетное подмножество. Первый элемент $b_0 \in A$ выберем произвольно. Затем рассматриваем множество $A \setminus \{b_0\}$. Оно бесконечно, а значит мы можем выбрать новый элемент $b_1$. Утверждается, что этот процесс мы можем проделать бесконечное количество раз: на каждом $i$ шаге мы остаемся с множеством $A \setminus \{b_o \ldots b_i\}$, которое бесконечно.
	\end{proof}
    
	\qed %ставьте кеду пожлауйста, так будет красивее немного
	\sep	
	
	
	\section*{Задача 2}
	
	\subsection*{Решение}
	
	\section*{Задача 3}
	
	
	\subsection*{Решение}
	
	
	\sep	
	
	\section*{Задача 4}

	\subsection*{Решение}


	\sep	
	
	\section*{Задача 5}
	
	\subsection*{Решение}
	
	

	\sep	
		
		
	\section*{Задача 6}

	
	\subsection*{Решение}
	


	\sep		
	
	
	
	
	\section*{Задача 7}

	
	\subsection*{Решение}
	


	\sep	
	
	
	
	
\end{document}
