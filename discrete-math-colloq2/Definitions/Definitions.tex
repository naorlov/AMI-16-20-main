\documentclass[a4paper,12pt]{article}

\usepackage{header}

\begin{document}
	\title{Дискретная математика. Коллоквиум весна 2017.\\ Определения}
	\author{Потом заполню}
	\maketitle
	
	%Определения даются просто как \item.
	\begin{enumerate}
		\item 6
		Множества называются \textit{равномощными}, если между ними существует \textit{биекция}, или взаимо-однозначное соответствие. Равномощность множеств обозначают значком $\sim$.
		
		Свойства равномощности:
		\begin{enumerate}
			\item \textit{Cимметричтность:} $A \sim B \Rightarrow B \sim A$.
			\item \textit{Рефлексивность:} $\forall A: \ A \sim A$
			\item \textit{Транзитивность:} $A \sim B, \ B \sim C  \Rightarrow A \sim C$
		\end{enumerate}
	
		\item 8
		Множество $S$ называют \textit{счетным}, если оно равномощно множеству $\N$.	Счетные множества обладают некоторыми свойствами:
		\begin{enumerate}
			\item Объединение счетных множеств счетно
			\item Всякое подмножество счетного множества конечно или счетно
			\item Всякое бесконечное множество содержит счетное подмножество
			\item Множество $\Q$ рациональных чисел счетно
			\item 
			Конечное либо счетное объединение конечных либо счетных множеств конечно либо счетно.
			
			\item Декартово произведение счетных множеств $A \times B$ счетно.
			\item Число слов в конечном или счетном алфавите счетно.
		\end{enumerate}
		
	\end{enumerate}
		
	
\end{document}
