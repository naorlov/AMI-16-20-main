\documentclass[a4paper,12pt]{article}

\usepackage{header}

\begin{document}
	\title{Дискретная математика. Коллоквиум весна 2017.\\ Определения}
	\author{Потом заполню}
	\maketitle
	
	%Определения даются просто как \item.
	\begin{enumerate}
		\item 6
		Множества называются \textit{равномощными}, если между ними существует \textit{биекция}, или взаимо-однозначное соответствие. Равномощность множеств обозначают значком $\sim$.
		
		Свойства равномощности:
		\begin{enumerate}
			\item \textit{Cимметричтность:} $A \sim B \Rightarrow B \sim A$.
			\item \textit{Рефлексивность:} $\forall A: \ A \sim A$
			\item \textit{Транзитивность:} $A \sim B, \ B \sim C  \Rightarrow A \sim C$
		\end{enumerate}
	
		\item 8
		Множество $S$ называют \textit{счетным}, если оно равномощно множеству $\N$.	Счетные множества обладают некоторыми свойствами:
		\begin{enumerate}
			\item Объединение счетных множеств счетно
			\item Всякое подмножество счетного множества конечно или счетно
			\item Всякое бесконечное множество содержит счетное подмножество
			\item Множество $\Q$ рациональных чисел счетно
			\item 
			Конечное либо счетное объединение конечных либо счетных множеств конечно либо счетно.
			
			\item Декартово произведение счетных множеств $A \times B$ счетно.
			\item Число слов в конечном или счетном алфавите счетно.
		\end{enumerate}
		\item 9 \\
		Континуум - мощность множества [0,1]. Примеры:
		\begin{enumerate}
			\item Множество бесконечных последовательностей нулей и единиц
			\item Множество вещественных чисел
			\item Квадрат [0,1]x[0,1].
		\end{enumerate}
		\item 10
		Свойства континуума:
		\begin{enumerate}
		\item В любом континуальном множестве есть счетное подмножество.
		\item Мощность объединения не более чем континуального семейства множеств,
		каждое из которых не более чем континуально, не превосходит континуума.
		\end{enumerate}
		
		\item 11
		Булева функция от $n$ аргументов - отображение из $B^{n}$ в $B$, где B - $\{0,1\}$.
		Количество всех $n$-арных булевых функций равно $2^{2^{n}}$. Булеву функцию можно задать таблицей истинности.
		
		\item 12
		Полный базис - это такой набор, который для реализации любой сколь
		угодно сложной логической функции не потребует использования каких-либо других
		операций, не входящих в этот набор.
		Примеры полных базисов:
		\begin {enumerate}
		\item Конъюнкция, дизъюнкция, отрицание.
		\item Конъюнкция, отрицание.
		\item Конъюнкция, сложение по модулю два, константа один - базис Жегалкина.
		\item Штрих Шеффера (таблица истинности - 0111).
		\end {enumerate}
		
		\item 24
		Универсальную вычислимую функцию $U(p, x)$ называют \textit{главной}, если для любой вычислимой функции $V(q, y)$ существует \textit{транслятор} -- вычислимая тотальная функция, такая что 
		\[
		\forall q, y: \ V(q, y) = U(s(q), y)
		\]
		Такие функции также называются \textit{главными нумерациями}.
		\item 25
		Пусть $F = \{ f \ | \ f : \N \rightarrow \N \}$ - множество вычислимых функций. Пусть $A \subseteq F$ - подмножество функций. Говорят, что функция удовлетворяет некому \textit{свойству}, если она лежит в $A$. Пусть $U(p, x)$ -- универсальная функция. Пусть $P_a = \{ p \ | \ U(p, x) \in A \}$. Утверждается, что если A -- нетривиально (т.е $A \neq \oslash, \overline{A} \neq \oslash$), то множество $P_a$ неразрешимо.
		
	\end{enumerate}
		
	
\end{document}
